\documentclass{article}
\usepackage[a4paper, total={6in, 8in}]{geometry}
\usepackage{lineno}

\title{Understanding the effect of date rounding for phylodynamics}
\author{Leo A. Featherstone$^{\ast,1}$, Courtney Lane$^{1}$ Ben Howden$^{1}$ Danielle Ingle$^{1}$ Sebastian Duchene$^{1}$\\}

\begin{document}

\begin{centering}
\maketitle
\end{centering}
$^{1}$Peter Doherty Institute for Infection and Immunity, University of Melbourne, Australia.\\
*email: leo.featherstone@unimelb.edu.au

\linenumbers
\section*{Abstract}
\section*{Methods}
\section*{Results}

\begin{table}[ht]
\centering
\caption{Absolute error between mean posterior estimates and true values for each epidemiological parameter and simulation scenario. Absolute error is measured as the mean difference between the true value for each parameter and the mean estimated value, averaged over each of the 100 replicates for each simulation scenario. Error increases as date resolution decreases (day $\rightarrow$ year), except for the under the tb scenario where estimates are overall stable and robust to date rounding.}
\label{table:sim_tab}
\begin{tabular}{llrrrrrrr}
  \hline
 organism & resolution & meanR0Err & meanRe1Err & meanRe2Err & meanPErr & meanDeltaErr & meanOriginErr \\ 
  \hline
    h1n1 & Day & 0.03 &  &  & 0.00 &  & 0.03 \\ 
    h1n1 & Month & 1.48 &  &  & 0.06 &  & 0.05 \\ 
    h1n1 & Year & $>10^8$ &  &  & 0.47 &  & 0.25 \\ 
    \hline
    sars-cov-2 & Day & 0.94 &  &  &  & 12.56 & 0.01 \\ 
    sars-cov-2 & Month & 39.78 &  &  &  & 33.12 & 0.02 \\ 
    sars-cov-2 & Year & 28.98 &  &  &  & 74.51 & 0.02 \\ 
    \hline
    shigella & Day &  & 0.42 & 0.05 & 0.25 &  & 0.03 \\ 
    shigella & Month &  & 0.37 & 0.62 & 0.45 &  & 0.07 \\ 
    shigella & Year &  & 0.23 & $>10^{12}$ & 0.28 &  & 0.46 \\ 
    \hline
    tb & Day &  & 0.54 & 0.20 & 0.02 & 0.56 & 2.17 \\ 
    tb & Month &  & 0.55 & 0.20 & 0.02 & 0.56 & 2.15 \\ 
    tb & Year &  & 0.57 & 0.19 & 0.02 & 0.60 & 2.18 \\ 
 \hline
\end{tabular}
\end{table}



\end{document}