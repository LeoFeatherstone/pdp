\documentclass[11pt]{article}

\usepackage[a4paper, total={6in, 8in}]{geometry}
\usepackage{setspace}
\usepackage[modulo]{lineno}
\usepackage[dvipsnames]{xcolor}
\usepackage{caption}
\captionsetup{font=small}
\usepackage{graphicx}
\usepackage{float}
\usepackage{booktabs}
\usepackage[square,sort,comma,numbers]{natbib}
\setcitestyle{authoryear,open={(},close={)}}

\graphicspath{{figures/}}
    
\title{How does date-rounding affect phylodynamic inference for public health?}
\author{Leo A. Featherstone$^{\ast,1}$, Wytamma Wirth$^{1}$,Courtney Lane$^{1}$, Benjamin Howden$^{1}$,\\Danielle Ingle$^{1}$, Sebastian Duchene$^{1}$}

\begin{document}

\maketitle

$^{1}$ Peter Doherty Institute for Infection and Immunity, University of Melbourne, Australia.\\
*email: leo.featherstone@unimelb.edu.au

\section*{Abstract}
 TODO: Leo to rewrite at end \\
 Despite its broad applicability, phylodynamic analyses always begin with pathogen genome sequences and associated sampling times. But, the release of sampling time data can pose a threat to patient confidentiality. In response, sampling times are often only given with reduced resolution to the month or year, but this can bias estimates and mislead the interpretation of results. Here, we characterise the extent to which reduced sampling time resolution can bias phylodynamic analyses across a diverse range of empirical and simulated datasets and offer a guideline on when date rounding biases phylodynamic inference. We also show that this bias is both unpredictable in its direction and compounds for increased evolutionary rates and decreased date-resolution. We conclude by suggesting a simple form of encryption where exact sampling times are essential and there is a clear risk to patient confidentiality.


\begin{spacing}{1.5}
\linenumbers

\section*{Introduction}

Phylodynamics is commonly use to estimate the parameters of epidemiological spread from pathogen genome sequences. It offers valuable insight where the age, origin, and transmission is difficult to infer from other data. 


Pathogen genomics is playing an increasingly important role in our understanding of infectious outbreaks in recent decades. It offers insight across the scales of transmission from the pandemic and epidemimec scales, such as for SARS-CoV-2 and Ebola viruses, to more localised transmission or bacterial pathogens \citep{lancet2021genomic}. Phylodynamic analysis has consequently emerged as a key method to make temporal and spatial inference from pathogen genomes, particularly since the 2013-2016 West African Ebola outbreak \citep{mbala2019medical}. It is most useful where temporal and spatial records of transmission are sparse, because it allows genomic information help fill the gaps of traditional epidemiological data.

Phylodynamic methods offer a range of inferences from pathogen sequence data, including epidemioloigcal transmission and migration rates, pathogen population size, spatial dynamics, time and location of origin, and the identification of variants of concern \citep{featherstone2022epidemiological, attwood2022phylogenetic, du2015getting,volz_fitness_2023}. The basis of all such inferences is that epidemiological spread leaves a trace in the form of substitutions that can be used to reconstruct transmission processes. Pathogen populations that meet this assumption known as `measurably evolving populations' \citep{drummond2003measurably, biek_measurably_2015}. In accordance, phylodynamics uses a combination of genome sequence and associated sampling times to leverage measurable evolution and infer temporally explicit parameters of transmission and pathogen demography.

Ideal phylodynamic datasets should include precise sampling dates alongside genome sequences \citep{black2020ten}, but sampling times necessarily carry over information about times of hospitalisation, testing, or treatment than can be used to identify individual patients. This can pose an unacceptable risk for patient confidentiality. In some cases, sampling times or dates of admission that are even available for purchase have allowed identification for a majority of patients on record \citep{sweeney_matching_2013,shean_private_2018}. In acknowledgement of this risk, \citet{shean_private_2018} suggest that Expert Determination govern whether sampling times be released alongside genome sequences, and the resolution to which they are disclosed (day, month, year). Essentially, This process involves an expert opinion on whether information is safe to release on a dataset-by-dataset basis.

From a phylodynamic point of view, sampling times with reduced resolution are usable. In principle, uncertainty in sampling times can be modelled, using Bayesian inference \citep{shapiro2011bayesian}, but such an approach is effective when samples with uncertain dates are a small proportion of the data set \citep{rieux2017tipdatingbeast}.

The most common technique for using imprecise sampling times requires the researcher to assume that they have been sampled at an arbitrary day. Selecting the arbitrary day can be motivated by convenience, such as all samples from 2020 being assigned 1st January 2020 or 15 June 2020, or by sampling a random day within 2020 using a statistical distribution for each sample. In either case, both approaches introduce a degree of error. Moreover, theoretical and applied results show sampling times can substantially direct inference \citep{featherstone_decoding_2023,featherstone_infectious_2021,volz_sampling_2014}, raising the question of whether reduced date resolution biases inference. This question has practical significance, as there are many examples of phylodynamic analysis conducted for a diverse array of viral and bacterial pathogens with reduced date resolution. These include viral pathogens such as Rabies Virus, Enterovirus, SARS-CoV-2, Dengue virus \citep{talbi_phylodynamics_2010,xiao_genomic_2022,wolf_temporal_2022,bennett_epidemic_2010}, and bacterial pathogens such as \textit{Klebsiella pneumoniae}, \textit{Streptococcus pneumoniae}, and \textit{Mycobacterium tuberculosis} \citep{cella_multi-drug_2017,azarian_impact_2018,merker_evolutionary_2015}. 

The degree of precision in collection times is also relevant for database design and curation because sampling dates are often considered part of the associated metadata and may be unavailable or imprecise due to inconsistencies between between storage platforms \citep{raza2016big}. For example, as of early August 2023, there were roughly 15.8M SARS-CoV-2 genome sequences available on GISAID with 2.4\% (382K) of these having `incomplete' date information, where sampling dates are absent or only given to the year. In other words, roughly 1 in 50 sequences lacked clear date resolution, reflecting global inconsistency in sampling times records.

In acknowledgment of this issue, we characterised when and how biases arise from reduced date resolution in phylodynamic inference. We analysed four empirical datasets of SARS-CoV-2, H1N1 Influenza, \textit{M. tuberculosis}, and \textit{Staphylococcus aureus} and conducted a simulation study with conditions parameters corresponding to each dataset. These pathogens have undergone substantial genome surveillance and have different infectious periods and molecular evolutionary dynamics, thus providing a diverse representation of phylodynamics' applicability. For each empirical and simulated dataset, we repeated analyses with sampling times rounded to the day, month, or year and studied the resulting bias in inferred parameters. For example, 2021/10/11 would be specified as 2021/10/15 when the day is not provided and 2021/06/15 when the month and day are not provided.

We focused on inference of the reproductive number ($R_0$ or $R_e$), defined as the average number of secondary infections stemming from an individual case, the age of each outbreak, and the substitution rate (per site per year) in each dataset. Together, these parameters span much of the insight that phylodynamics offers through inferring when an outbreak started and how fast it proceeded. The evolutionary rate is also the central parameter relating evolutionary time to epidemiological time, so any resulting bias has a pervasive effect throughout a phylodynamic model.

 We hypothesise that reduced date resolution causes bias that compounds where the temporal resolution that is lost exceeds the average time required for a substitution to arise in a given pathogen. We visualise the relationship between date resolution and average substitution time in Fig \ref{fig:plane}. For example, H1N1 influenza virus accumulates substitutions at a rate of about 4 $\times10^{-3}$ subs/site/year \citep{hedge_2013_real-time}. With a genome length of 13,158bp, we then expect a substitution to accrue roughly every week. Therefore, rounding dates to the month or year will conflate the times over which substitutions arise and bias the resulting inferences. 
 
 Our results across the simulation study and analyses of empirical data support using the average substitution time as a rough guideline for where date rounding causes excessive bias. We also discuss factors that modulate the extent of bias such as duration of sampling intervals and the choice of phylodynamic model. We finish by discussing future solutions that prioritise both patient confidentiality and accurate data sharing for routine phylodynamic analyses for public health.

\begin{figure}[H]
    \centering
    \includegraphics[width = 0.5\textwidth]{plane.pdf}
    \caption{The average time to accrue one substitution based on a fixed genome size and evolutionary rate ($\frac{\textnormal{Genome Length}}{\textnormal{Evolutionary rate (subs/site/time)}}$) against the temporal resolution lost by date rounding. We hypothesised and showed that when analyses for a given pathogen round dates to an extent nearing or crossing the diagonal from left to right, biases is induced in $R_e$, age of the outbreak, and evolutionary rate. substitution rates are taken from each source for the empirical data. We do not report the numerical axis as this figure is designed to illustrate a concept rather than serve as a reference, in the same spirit as is inspiration in Figure 2 of \citet{biek_measurably_2015}.}
    \label{fig:plane}
\end{figure}


\section*{Methods}
\subsection*{Overview}
Our study is based on 4 empirical datasets of H1N1 influenza virus, SARS-CoV-2, \textit{Staphylococcus aureus}, and \textit{Mycobacterium tuberculosis}. We also conducted a  simulation study with parameters tailored for each dataset. These data were chosen to capture span the usual parameter space for substitution rate and sampling duration in phylodynamics for epidemiology (roughly $10^{-3}\textnormal{-to-}10^{-8}$ (subs/site/yr) for substitution rate and Months-to-Decades for duration of sampling).

To assess the effects of date rounding, we conducted phylodynamic analyses for both the empirical and simulated datasets with sampling dates rounded to the day, month, or year. For example, two samples from 2000/05/29 and 2000/05/02 would become 2000/05/15, if rounded to the month. We then measured the resulting bias in epidemiologically- or phylodynamically-important parameters: the reproductive number ($R_0$ or $R_e$), substitution rate (subs/site/year), and the age of the outbreak (time to most recent common ancestor, tMRCA hereon). 

In this context $R_0$ refers to the \textit{basic} reproductive number and $R_e$ is the \textit{effective} reproductive number. These parameters correspond to the average number of secondary infections at the start of an outbreak (i.e. assuming a fully susceptible population; $R_0$) or thereafter ($R_e$) (reviewed by \citep{featherstone2022epidemiological, du2015getting, kuhnert2011phylogenetic}). We consider the age of outbreaks as the tMRCA to facilitate comparison between phylodynamic models because some analyses include the length of the root branch in the age of the outbreak \citep{stadler2012estimating}.

The two viral datasets consist of samples from the 2009 H1N1 pandemic (n=161) from \citet{hedge_2013_real-time}, and a cluster of early SARS-CoV-2 cases from  Victoria, Australia in 2020 (n = 112) \citep{lane2021genomics}. The bacterial datasets consist \textit{S. aureus} 104 samples from New-York sampled over ~2 years \citet{duchene_2016_genome,volz_modeling_2018,uhlemann_molecular_2014}, and 30 \textit{M. tuberculosis} samples from a ~25 year outbreak studied by \citet{kuhnert_tuberculosis_2018}. These data were chosen because they encompass a diversity of epidemiological dynamics, timescales, and variable rates of substitution.

\subsection*{Simulation Study}

\begin{table}[H]
    \centering
    \caption{Parameter sets outbreaks corresponding to each empirical dataset.}
    \begin{tabular}{l|c|c|c|c|c|c|l|}
    \hline
    Microbe                     &   $\delta (yrs)^{-1}$    & $R_0$ &   $R_{e_1}$   &  $R_{e_2}$    &   $p$   &   $T$(yrs)   & Source \\
    \hline
    H1N1                        &   91.31    & 1.3 &   -   &  -    &   0.015   &   0.25 & \citet{hedge_2013_real-time} \\
    SARS-CoV-2                  &   36.56    & 2.5 &   -   &  -   &   0.80   &  0.16 & \citet{lane2021genomics} \\
    \textit{S. aureus}    &   0.93    &  - &   2.0   &  1.0   &   0.2$^{\dagger}$   &   25 & \citet{duchene_2016_genome} \\
    \textit{M. tuberculosis}    &   0.125    &  - &   2.0   &  1.10    &   0.08   &   25.0 & \citet{kuhnert_tuberculosis_2018} \\
    \hline
    \end{tabular}
    \label{tab:sim_parms}
\end{table}
\footnotesize{$^\dagger$ $p$ was set to zero before $T=22$}

We simulated outbreaks as birth-death sampling processes using the ReMaster package in BEAST v2.7.6 \citep{vaughan_remaster_2024,bouckaert_beast_2019}. Simulations consisted of 4 parameter settings corresponding to each empirical dataset (Table \ref{fig:emp-parm}), with 100 replicates over each. All parameter sets include a proportion of sequenced cases ($p$), outbreak duration ($T$), and a "becoming un-infectious" rate ($\delta$ = reciprocal of the duration of infection). For simulations corresponding the viral datasets, transmission is modelled via $R_0$, the average number of secondary infections (assuming a fully susceptible population). For those corresponding to the bacterial datasets, we allow the effective reproductive numbers to vary over two intervals ($R_{e_1}$ and $R_{e_2}$ respectively). For the \textit{S. aureus} setting, the change time was at $t=22$ with the sequencing proportion ($p$) also set to zero before this time to replicate the sampling effort generating the dataset. For the \textit{M. tuberculosis} dataset, the change time was fixed at halfway through simulations ($t=12.5$) with one sequencing proportion throughout.

\begin{table}[H]
    \centering
    \caption{Substitution rates and genome length for sequence simulation.}
    \begin{tabular}{l|c|l|l}
    \hline
    Microbe                     &   Substitution Rate (subs/site/yr) & Genome Length & Time/Sub/Genome (yrs)  \\
    \hline
    H1N1                        & $4\times10^{-3}$ & 13158 & 0.0190\\
    SARS-CoV-2                  & $1\times10^{-3}$ & 29903 & 0.0334\\
    \textit{S. aureus}    & $1\times10^{-6}$ & 2900000  & 0.3458\\
    \textit{M. tuberculosis}    &   $1\times10^{-7}$ & 4300000 & 2.3256\\
    \hline
    \end{tabular}
    \label{tab:seq_parms}
\end{table}

Overall, simulations yielded a total of 400 outbreaks which we then used to simulate sequences data under a Jukes-Cantor model using Seq-Gen v1.3.4 \citep{rambaut_seq-gen_1997} with fixed substitution rates (Table \ref{tab:seq_parms}). We chose a simple substitution model to reduce parameter space and because substitution model mismatch has been widely explored elsewhere \citep{lemmon2004importance}.

We then analysed each of the 400 simulated datasets under each of the 3 date treatments (Day, Month, and Year resolution) and yielding 1800 analyses (1200 under birth death and 600 under coalescent exponential tree-priors). We used identical model specifications and prior setting as for the corresponding empirical datasets. We ran each MCMC chain for $5\times18^{8}$ steps, sampling every $10^{4\textnormal{th}}$ step and discarding the first 50\% as burnin. We then discarded all analyses that did not have $ESS\geq200$, leaving a total of 1670 joint posteriors.

\subsection*{Empirical Data}
We conducted Bayesian phylodynamic analyses using a birth-death skyline tree-prior in BEAST v2.7.6 for all datasets \citep{bouckaert_beast_2019,stadler2012estimating}. We also fit a coalescent exponential tree-prior for the viral datasets \citep{kingman_1982_coalescent}. We sampled from the posterior distribution using Markov chain Monte Carlo (MCMC), with $5\times10^{7}$ steps ($1\times10^{7}$ for SARS-CoV-2 data) sampled every $10^{4}$, and the initial 10\% discarded as burnin. We assessed convergence by ensuring $ESS\geq200$ for all parameters and likelihoods.

\subsubsection*{H1N1}
The H1N1 data consist of 161 samples from North America during the 2009 H1N1 influenza virus pandemic, analysed by \citet{hedge_2013_real-time}. Samples originate from April to September 2009 and provides an example of a rapidly evolving pathogen sparsely sampled during an emerging outbreak. 

Under the birth-death, we placed a $\textnormal{Lognormal}(\mu=0,\sigma=1)$ prior on $R_0$, $\beta(1,1)$ prior on $p$, and fixed the becoming-uninfectious ($\delta = 91$), corresponding to a 4 day duration of infection. We also placed an improper ($U(0,\infty)$) prior on the age of the outbreak and a $\textnormal{Gamma}(\textnormal{shape}=2,\textnormal{rate}=400)$ prior on the substitution rate.

Under the coalescent exponential, we placed a $\textnormal{Laplace}(\mu=0,\textnormal{scale}=100)$ prior on the growth rate, which was later transformed to $R_0$ via $R_0 = rD+1$ where $r$ is the growth rate and $D$ is the duration of infection. We also placed an improper prior on the effective population size, with other priors matching those under the birth-death.

\subsubsection*{SARS-CoV-2}
The SARS-CoV-2 data are 112 samples from a densely sequenced transmission cluster in Victoria, Australia in from late July to mid September 2020 \citet{lane2021genomics}. These data are similar to the H1N1 datasets in presenting a quickly evolving viral pathogen, but contrast in that virtually all cases in the cluster were sequenced.

Under the birth-death, we placed a $\textrm{Lognormal}(\textrm{mean}=1, \textrm{sd}=1.25)$ prior on $R_0$ and an $\textrm{Inv-Gamma}(\alpha=5.807, \beta=346.020)$ prior on the becoming-uninfectious rate ($\delta$).  The sampling proportion was fixed to $p=0.8$ since every known Victorian SARS-CoV-2 case was sequenced at this stage of the pandemic, with a roughly 20\% sequencing failure rate. We also placed an $\textrm{Exp}(\textrm{mean}=0.019)$ prior on the origin, corresponding to a lag of up to one week  between the index case and the first putative transmission event. Lastly, we placed a $\textnormal{Gamma}(\textnormal{shape}=2,\textnormal{rate}=2000)$ prior on the substitution rate.

Under the coalescent exponential, we placed an improper prior on the effective population size and a $\textnormal{Laplace}(\mu=0.01,\textnormal{scale}=0.5)$ prior on the growth rate. Other parameters also present under the birth-death were given the same priors.


\subsubsection*{\textit{Staphylococcus aureus}}
The \textit{S. aureus} dataset originates from \citet{uhlemann_molecular_2014} and we analysed a subset of the data later considered in \citet{duchene_2016_genome} and \citet{volz_modeling_2018}. It consists of a single nucleotide polymorphism (SNP) alignment of 104 sequenced isolates sampled in New York from 2009 to 2011. Populations growth is understood to have been driven by $\beta$-lactam use beginning in the 1980s. These data therefore provide a comparison to the \textit{M. tuberculosis} dataset in a shorter sampling span from am outbreak of similar duration.

To accommodate changing transmission dynamics, we included two intervals for $R_e$ with a $\textnormal{Lognormal}(\mu=0,\sigma=1)$ prior on each. We also placed a $\beta(1,1)$ prior on the sampling proportion, which was otherwise fixed to 0 before the first sample to capture the lag in sampling. We also placed a $U(0,1000)$ prior on the origin, and fixed the becoming un-infectious rate at $\delta=0.93$, corresponding to a nearly year-long duration of infection following \citet{volz_modeling_2018}.

\subsubsection*{\textit{Mycobacterium tuberculosis}}
The \textit{M. tuberculosis} dataset consists of 36 sequenced isolates from a retrospectively recognised outbreak in California, USA, that originated in the Wat Tham Krabok refugee camp in Thailand and was originally studied in \citet{kuhnert_tuberculosis_2018}. We applied the same prior configurations as \citet{kuhnert_tuberculosis_2018}, with the exception of including 2 intervals for $R_e$ and fitting a strict molecular clock with a $\textnormal{Gamma}(\textnormal{shape}=0.001,\textnormal{rate}=1000.0)$ prior.

\section*{Results}
\subsection*{Simulation study}
The viral simulation conditions (SARS-CoV-2 and H1N1) display the greatest bias in mean posterior estimates of substitution rate, outbreak age, and reproductive number with decreasing date resolution (Figure \ref{fig:sim-parms} A-C). The bacterial simulation conditions exhibit  similar trend with lesser lesser bias in response to decreasing date resolution. The \textit{M. tuberculosis} condition is effectively inert to decreasing date resolution, with mean posterior estimates for each parameter of interest remaining consistent across date resolution (day to year). The \textit{S. aureus} provides an important intermediate case in that estimates of each parameter change when transitioning from month to year resolution (see crossing of lines from Month to Date resolution in the \textit{S. aureus} column of Figure \ref{fig:sim-parms}). These trends are in agreement with the hypothesis of decreasing date resolution causing increased bias in parameters of interest where the resolution lost exceeds the average time for a substitution to arise. This occurs because date rounding compresses divergent sequences in time, driving a signal for higher rates of substitution and transmission locally to each temporal cluster of sampling times.  The effect is less pronounced in the bacterial conditions where the date resolution lost is a smaller fraction of the effective substitution time, such as for the \textit{M. tuberculosis} and \textit{S. aureus} conditions (average time to until substitution is $\approx4$ months and $\approx28$ months respectively, Table \ref{tab:seq_parms}). However, there are also notable deviations from these general trends across date resolutions and simulation conditions that we attribute to the duration of the sampling intervals below.

\begin{figure}[H]
    \centering
    \includegraphics[width=\textwidth]{simulation_parm_panel.pdf}
    \caption{Mean posterior estimates for parameters of interest for each simulated dataset varying across date resolution. Individual lines track mean posterior estimates for each simulated dataset and boxplots are given to summarise the spread and direction of bias across all simulated datasets for each date resolution. Rows correspond to individual parameters, columns correspond to simulation conditions (underlying parameters matching each empirical dataset), and colour corresponds to tree prior or reproductive number interval. Dashed horizontal lines correspond to the true value under which each dataset was simulated. (\textbf{A}) Mean posterior substitution rate across simulation scenarios. (\textbf{B}) Mean posterior age of outbreaks (tMRCA). (\textbf{C}) mean posterior reproductive number. }
    \label{fig:sim-parms}
\end{figure}

The coalescent exponential shows overall downwards bias in the substitution rate for the SARS-CoV-2 and H1N1 treatments at month resolution, while the birth-death exhibits upwards bias. Since the sampling times for each viral dataset are distributed over 3 months, date rounding compresses samples within a month to one time while also increasing the time between samples across months, driving a signal for lower transmission and substitution rates between months. The different phylodynamic likelihood functions for each tree prior appear to respond to this warped distribution of diversity over time with the coalescent exponential placing more weight on decreased rates of evolution while the birth death places more on increased substitution rates. This can be explained by the birth death drawing signal for increased transmission among coincident sampling times within each month, while the coalescent exponential instead conditions on these \citep{volz_sampling_2014}. At year-resolution there is there is also lower bias in estimates of substitution rate for the coalescent exponential than the birth-death, however both models prefer increased substitution rates as year resolution. This is probably because year resolution clusters all sampling times to one, meaning a highly inflated rate of substitution is needed to model the artificial burst in diversity at one time point that date rounding implies (See Figure \ref{fig:densitree} to see sampling times compressed in time across date resolution). For all viral simulation conditions, the mean posterior age of each outbreak shifts inversely to the substitution rate. This is the result of a well understood axis among phylodynamic models where higher rates of evolution suggest shorter periods of evolution.

The reproductive number for each viral dataset ($R_0$) also changes markedly with decreasing date resolution under the birth death but not under the coalescent. For the birth-death, this is in agreement with temporal clustering of samples driving a signal for higher transmission rates. Conversely, estimates under the coalescent exponential remain largely unchanged at Month-resolution. This is probably because the model can adjust the population size parameter, which is nor reflected in posterior $R_0$ instead of modulating the growth rate. Estimates of $R_0$ for the SARS-CoV-2 condition are also heavily biased downwards. This is probably due to high sequencing proportions violating the assumption of low sampling under the coalescent, thus leading to poorly fitting model in the first place.

The \textit{S. aureus} condition yields consistent estimates of substitution rate, outbreak age, and reproductive number ($R_e$ in this case) when dates are rounded to the month (Figure \ref{fig:sim-parms} \textit{S. aureus} column). At year resolution the posterior substitution rate appears biased downwards. This can be explained by the two year sampling duration of the \textit{S. aureus} condition, such that samples rounded to the year will be on average further apart in time than if dates are given to the month or day (Figure \ref{fig:densitree}). This spacing of diversity in time likely drives the signal for lower substitution rates and an older outbreak in turn. There is no clear pattern in the direction of bias for $R_{e_1}$ and $R_{e_2}$ at year resolution, though estimates deviate from those at month and day resolution. Estimates for $R_{e_1}$ are also overall lower than their true value of 2, and this is attributable to inconsistent sampling over the duration of the outbreak which was previously demonstrated for other datasets with late sampling in \cite{featherstone_infectious_2021}.

The \textit{M. tuberculosis} simulation condition effectively acts as a control, since it appears inert to date rounding. This is expected because this dataset reflects both longer simulation time, with temporal clustering less likely to inflate $R_e$, but also the effective substitution time is longer than 1 year. As such, even rounding to the year is unlikely to drive a signal for increased evolutionary rate or a more recent origin time.

Phylodynamic and phylodynamic likelihoods also vary with decreasing date resolution (Figure \ref{fig:sim-likeihood}). Variation also increases with lesser date resolution between Month and Year. This shows that altered date resolution affects the likelihood manifold of each analysis, which is reflected in the different trends of bias in each parameter of interest.

\subsection*{Empirical Results}
Broadly, analyses of the empirical datasets reproduce the patterns of bias in the simulation study(Figure \ref{fig:emp-parm}). That is, the reproductive number increases with decreasing date resolution along with an increase in the substitution rate and corresponding decrease in the age of outbreak. There are a few exceptions to this trend that we consider below and which we attribute to the difference between simulated and empirical sampling time distributions.

\begin{figure}[H]
    \centering
    \includegraphics[width=\textwidth]{empirical_parms.pdf}
    \caption{Posterior distributions for parameters of interest estimated for each empirical dataset. Date resolution is given on the horizontal axis and colour denotes tree-prior. Estimates for viral datasets at year-resolution are omitted because results deviate by implausible orders of magnitude due to sampling times rounded to identical dates. (\textbf{A}) Posterior substitution rate across date resolutions. (\textbf{B}) Posterior outbreak age (tMRCA) in units of months (m) or years (y).  (\textbf{C}) Posterior reproductive number on a log-transformed axis.}
    \label{fig:emp-parm}
\end{figure}

Phylodynamic and phylogenetic likelihoods also diverge where the loss in date resolution exceed the average time for a substitution to arise (Figure \ref{fig:emp-likelihood}). For both viral datasets, month and day posterior distributions of phylodynamic and phylogenetic likelihood are diverged, while all are overlapping for the \textit{M. tuberculosis} data. The \textit{S. aureus} data provide an intermediate case where only the posterior likelihoods for year-resolution differ. Together, these likelihood distributions support the hypothesis that date resolution larger than the average time to one substitution cause a shift in the likelihood manifold for analyses under both birth-death based and coalescent tree priors.

\begin{figure}[H]
    \centering
    \includegraphics[width=\textwidth]{figures/empirical_likelihood.pdf}
    \caption{Posterior phylodynamic likelihood against phylogenetic likelihood for each combination of empirical dataset and with colour corresponding to date resolution. Ellipses represent the 95\% highest posterior density region. Both Phylodynamic and phylogenetic likelihoods diverge between day and month resolution for the viral datasets, while year resolution differs from month and day for the \textit{S. aureus} data. Posterior likelihoods all coincide for \textit{M. tuberculosis}. }
    \label{fig:emp-likelihood}
\end{figure}

\subsubsection*{H1N1}
Mean posterior $R_0$ increases from day to month resolution for the birth death (1.08 to 1.14), yet remains near-identical for the coalescent exponential (1.14 to 1.13) (Table \ref{tab:emp-ests}). The mean posterior substitution rate also decreases for both tree priors across day to month resolution ($4.3\times10^{-3}$ to $3.9\times10^{-3}$ and $3.9\times10^{-3}$ to $3.2\times10^{-3}$ for the birth death and coalescent exponential respectively) (Table \ref{tab:emp-ests}). The posterior age of outbreak also differs between tree priors mirroring substitution rate, with a decrease from date to month resolution for the birth death and an increase for the coalescent exponential. For the coalescent exponential, we can attribute the decrease in reproductive number and substitution rate from day to month resolution to more spread-out the samples further in time (Figure \ref{fig:sampling}), which drives the signal for and an older outbreak with lower transmission rates. While the same is true for the sampling distribution under the birth-death, the additional information it draws from identical sampling times as month-resolution likely inflates the mean posterior reproductive number while still driving a lower substitution rate and outbreak age.

\subsubsection*{SARS-CoV-2}
Under the birth death, the SARS-CoV-2 dataset behaves as expected with an increase in posterior $R_0$ from day to month rounding. In particular, rounding to the month results in an unlikely, but plausible value of $R_0 = 5.972$ (Table \ref{tab:emp-ests}). Under the coalescent exponential, mean posterior $R_0$ remains near identical across day to month treatments ($1.00$ to $1.01$ respectively). We note that the coalescent is included for completeness for the SARS-CoV-2 dataset, but is not an appropriate choice of model in practice due to the near complete-sequencing of the original transmission cluster. Thus estimates of $R_0$ under the coalescent exponential appear un-realistic and inert to in date rounding due to poorly fitting model.

The mean posterior substitution rate under the birth-death increases over two-fold when rounding to the month ($2.47\times10^{-4}$ to $6.56\times10^{-4}$, Table \ref{tab:emp-ests}). Mean posterior outbreak age also increases from 0.15 years to 0.17 years from day to month, which contradicts the expectation of a decreased estimate of outbreak age under date rounding. We again attribute these differences to the distribution of the empirical sampling times under date rounding. Sampling for the SARS-CoV-2 dataset mainly occurred over August to September 2020, with most August samples originating later in the month (Figure \ref{fig:sampling}). Rounding to 15$^{th}$ of August therefore made these samples appear older in time and likely contributed to the older origin under month-rounding.

\subsubsection*{\textit{S. aureus}}
For $R_{e_1}$, the \textit{S. aureus} dataset recapitulated the simulation study with month rounding having a minimal effect, but year rounding inducing an upwards bias (mean values of 1.57, 1.56, 1.73 respectively)(Figure \ref{fig:emp-parm}, Table \ref{tab:emp-ests}). $R_{e_2}$ presents the reverse pattern with consistent estimates at day and month-rounding before a reduction at year rounding (0.66, 0.67, and 0.37 respectively). This result is consistent with the estimates of ancestral growth rate in earlier analyses of the dataset \cite{volz_modeling_2018}.

Mean posterior substitution rate and outbreak age remain identical across date resolutions ($10^{-5}$ subs/site/year and an outbreak age of 30 year), despite the change in reproductive numbers at year rounding. This is surprising given the change in posterior phylodynamic and phylogenetic likelihoods (Figure \ref{fig:emp-likelihood}), and highlights that date-rounding can perturb likelihoods for inference without predictable changes in parameters of epidemiological significance.

\subsubsection*{\textit{M. tuberculosis}}
The \textit{M. tuberculosis} data recapitulate the outcome of the simulation study in being invariant to date rounding. Posterior substitution rates and outbreak ages remain consistent across decreasing date resolution ($1.02\times10^{-7}$, $1.02\times10^{-7}$, $9.86\times10^{-8}$ (subs/site/time) and 21.7, 21.7, and 22.5 years respectively) (Table \ref{tab:emp-ests}, Figure \ref{fig:emp-parm}). We also infer that $R_{e_1} > R_{e_s}$ across date-rounding conditions, coinciding with an earlier burst of transmission in agreement with \citet{kuhnert_tuberculosis_2018}. However, $R_{e_1}$ decreases slightly date date rounding (mean posterior estimates of $2.77$, $2.74$, $2.66$ for day, month and year rounding)(Table \ref{tab:emp-ests}) while $R_{e_2}$ increases (1.4, 1.41, 1.53 from day to year rounding). This was likely caused by the higher number of samples in the second sampling interval, from roughly 2002 to 2010, such that compressing sampling times drive drove a signal for higher transmission in the second interval with longer periods between sampling in the first interval at year resolution. Again, this shows that empirical sampling time distributions modulate the effects of date rounding.

\section*{Discussion}
The results of the simulation study and analyses of empirical data support our hypothesis that phylodynamic inference is most biased where the temporal resolution lost in rounding dates exceeds the average time for one substitution to arise. In the both the simulation study and empirical analyses, the viral datasets (H1N1 and SARS-CoV-2) display the greatest bias in mean posterior reproductive number, substitution rate, and age of outbreak when rounding to the month or year, with the average substitution time being less than one month in both simulation condition. The \textit{S. aureus} data provide an intermediate case, with estimated parameters displaying bias when rounding dates to the year (average substitution time between the order of months to a year). Lastly, the \textit{M. tuberculosis} data also provide supporting evidence in not displaying any notable bias between estimates at day, month, or year date rounding. This is expected because the average substitution time is less longer than a year in all \textit{M. tuberculosis} analyses.

We therefore propose the average substitution time as a rough practical threshold after which genomic epidemiologists can expect date rounding to introduce the strongest bias. However, we emphasise that this is without rigorous derivation and any degree of date rounding can alter likelihood and parameter estimation in phylodynamic analyses. Other factors such as as the length of the sampling window, distribution of sampling times over this interval, and choice of tree prior also affect the direction and severity of bias when rounding dates.

Shorter sampling intervals can exacerbate the bias due to date rounding. For example, in the SARS-CoV-2 data and simulation conditions, most samples originated over one month with the remainder towards the end of preceding months. Bias for these data was greater for each parameter compared to otherwise similar H1N1 data, which had a more even distribution of sampling over three full months. This result is in line with results previous results for ancient DNA data showing that date rounding has negligible effects for timescales of millennia or longer \citep{molak_2013_phylogenetic}. This emphasises the importance of accurate date for phylodynamic datasets of emerging pathogens, where results are likely to be the most urgent and reflect shorter sampling intervals.

The choice of tree-prior also affects bias when rounding dates. For example, the coalescent exponential tended to infer decreased substitution rates while the birth-death favoured increased substitution rates across simulated and empirical viral data. The inverse trend also arose for the outbreak age. This is because the birth death draws additional information from clustered sampling times, which serves to elevate rates of substitution and transmission, while the coalescent conditions on these and relies more on the longer duration between sampling times at month resolution for both datasets. 

Taken together, the results form the simulation study and empirical data show that although date rounding biases epidemiological estimates in a theoretically predictable directions, the intensity of the bias is difficult to predict and varies with the distribution and span of sampling times as well as tree prior. We conclude that accurate sampling time information is essential where phylodynamic insight is needed to understand infectious disease epidemiology and evolution. There does not appear to be an clear way to adjust for the bias caused otherwise. 

We also acknowledge that while accurate sampling times are essential for reliable phylodynamic results, it may pose an unacceptable level of risk to patient confidentiality to release sampling times. We therefore finish by disucssing potential future solution to this problem that prioritise both patient safety and data sharing.

\subsection*{Translating dates by random seeds}
The functional component of phylodynamic data are the \emph{differences} among genome sequences and among dates, rather than their absolute values. It may be possible to protect patient confidentiality by sharing the relative time between samples rather than their absolute time. For example, if the sampling times of a dataset of 3 samples are 2000, 2001 and 2002, then we may randomly draw a seed of 1000 with which to shift and dates, that is 2000, 2001 and 2002 $\rightarrow$ 3000, 3001 and 3002. Then results can be reinterpreted with regard to the random seed. For example if the estimated age of the outbreak was 5 years before the most recent sample, then the originating institution can privately date the outbreaks onset as 1997 (2002 - 5), while those conducting the analysis externally can only estimate the relative age of 5 years. In the same way, intervals of transmission parameters such as $R_e$ can be placed with respect to the true sampling times.

\subsection*{Distributed computing}
Approaches based on distributed computing where data are analysed at secure remote servers may also offer promise for optimising data quality and patient confidentiality. For example, \citet{santos_private_2022} recently developed a method to estimate phylogenetic trees using distributed computing and quantum cartographic protocols from private genome data. Routine phylodynamic analysis for genomic surveillance may also benefit from adopting protocols from socalled swarm learning approaches that allow artificial intelligence models in precision medicine to be trained across distributed datasets comprising a swarm \citep{warnat-herresthal_swarm_2021}.

\end{spacing}

\bibliographystyle{natbib}
\bibliography{refs}

\subsection*{Supplementary Material}

\renewcommand{\thefigure}{S\arabic{figure}}
\renewcommand{\thetable}{S\arabic{table}}
\setcounter{figure}{0}
\setcounter{table}{0}

\begin{figure}[H]
    \centering
    \includegraphics[width=\textwidth]{figures/empirical_sampling_times.pdf}
    \caption{The number of samples over time for each empirical dataset. Date rounding has the effect of moving each sampling within a month or year to the middle of that month or year ($15^{th}$ of the month or June $15^{th}$ of the year).}
    \label{fig:sampling}
\end{figure}

\begin{figure}[H]
    \centering
    \includegraphics[width=\textwidth]{empirical_densitrees.pdf}
    \caption{Desnsitrees (overlayed posterior trees) for empirical data with columns corresponding to pathogen under each combination of date resolution and tree prior. For the H1N1 and SARS-CoV-2 treatments, Year resolution causes trees to collapse to instantaneous bursts.}
    \label{fig:densitree}
\end{figure}

\begin{figure}[H]
    \centering
    \includegraphics[width=\textwidth]{simulation_likelihood.pdf}
    \caption{Adjusted phylodynamic likelihood against adjusted phylogenetic likelihood with panels corresponding to each simulation condition. Points correspond to mean posterior likelihood for each simulated dataset under each simulation condition. Colour corresponds to date resolution. Likelihoods are adjusted by subtracting the mean phylodynamic or phylogenetic likelihood at Day resolution from each the means under Month and year resolution. Resulting points therefore show the difference phylodynamic and phylogenetic likelihoods due to date rounding with the point $(0, 0)$ representing likelihood at day resolution for each dataset. Month resolution generally results in smaller differences that Year resolution, suggesting coarser date resolution results in more perturbed likelihoods. There is also generally more error in phylodynamic likelihood than phylogenetic likelihood.}
    \label{fig:sim-likeihood}
\end{figure}


\begin{table}[H]
    \centering
    \small
    \caption{Mean posterior estimates of substitution rate and outbreak age for empirical data with 95\% HPD in brackets.  The lower table gives mean posterior estimates of $R_{\bullet}$ for empirical data with 95\% HPD in brackets.}
    \begin{tabular}{lllcc}
         & Tree Prior & Resolution & Substitution Rate (subs/site/yr) & Outbreak Age\\
        \midrule
        H1N1 & BD & Day & 4.31e-3 (3.7e-3, 4.9e-3) & 3.67e-1 (3.3e-1, 4.2e-1)\\
        H1N1 & BD & Month & 3.9e-3 (3.2e-3, 4.6e-3) & 3.53e-1 (3.1e-1, 4.1e-1)\\
        H1N1 & CE & Day & 3.87e-3 (3.2e-3, 4.5e-3) & 4.25e-1 (3.6e-1, 5.0e-1)\\
        H1N1 & CE & Month & 3.17e-3 (2.6e-3, 3.8e-3) & 4.59e-1 (3.8e-1, 5.6e-1)\\
        SARS-CoV-2 & BD & Day & 2.47e-4 (1.1e-4, 4.5e-4) & 1.45e-1 (1.4e-1, 1.5e-1)\\
        \addlinespace
        SARS-CoV-2 & BD & Month & 6.56e-4 (3.3e-4, 1.1e-3) & 1.7e-1 (1.7e-1, 1.7e-1)\\
        SARS-CoV-2 & CE & Day & 2.37e-4 (9.1e-5, 4.7e-4) & 2.03e-1 (1.4e-1, 3.6e-1)\\
        SARS-CoV-2 & CE & Month & 4.34e-5 (4.4e-6, 1.4e-4) & 1.6 (2.9e-1, 5.9)\\
        \textit{S. aureus} & BD & Day & 1e-5 (1e-5, 1e-5) & 3e+01 (3e+01, 3e+01)\\
        \textit{S. aureus} & BD & Month & 1e-5 (1e-5, 1e-5) & 3e+01 (3e+01, 3e+01)\\
        \addlinespace
        \textit{S. aureus} & BD & Year & 1e-5 (1e-5, 1e-5) & 3e+01 (3e+01, 3e+01)\\
        \textit{M. tuberculosis} & BD & Day & 1.02e-7 (6.5e-8, 1.4e-7) & 2.17e+01 (1.7e+01, 3.2e+01)\\
        \textit{M. tuberculosis} & BD & Month & 1.02e-7 (6.6e-8, 1.4e-7) & 2.17e+01 (1.7e+01, 3.2e+01)\\
        \textit{M. tuberculosis} & BD & Year & 9.86e-8 (6.2e-8, 1.4e-7) & 2.25e+01 (1.8e+01, 3.4e+01)\\
\bottomrule
    \end{tabular}

    \label{tab:emp-ests}
\end{table}

\begin{table}[H]
    \ContinuedFloat
    \centering
    \small
    \begin{tabular}{lllccc}
     & Tree Prior & Resolution & $R_0$ & $R_{e_1}$ & $R_{e_2}$\\
    \midrule
    H1N1 & BD & Day & 1.08 (1.1, 1.1) & - & -\\
    H1N1 & BD & Month & 1.14 (1.1, 1.2) & - & -\\
    H1N1 & CE & Day & 1.14 (1.1, 1.2) & - & -\\
    H1N1 & CE & Month & 1.13 (1.1, 1.2) & - & -\\
    SARS-CoV-2 & BD & Day & 1.2 (9.3e-1, 1.6) & - & -\\
    \addlinespace
    SARS-CoV-2 & BD & Month & 5.85 (3.7, 9.0) & - & -\\
    SARS-CoV-2 & CE & Day & 1 (9.6e-1, 1.0) & - & -\\
    SARS-CoV-2 & CE & Month & 1.01 (9.8e-1, 1.1) & - & -\\
    \textit{S. aureus} & BD & Day & - & 1.57 (1.5, 1.7) & 6.56e-1 (5.1e-1, 8.0e-1)\\
    \textit{S. aureus} & BD & Month & - & 1.56 (1.5, 1.7) & 6.78e-1 (5.4e-1, 8.3e-1)\\
    \addlinespace
    \textit{S. aureus} & BD & Year & - & 1.73 (1.6, 1.8) & 3.71e-1 (1.9e-1, 5.4e-1)\\
    \textit{M. tuberculosis} & BD & Day & - & 2.77 (5.8e-1, 5.3) & 1.4 (7.2e-1, 2.7)\\
    \textit{M. tuberculosis} & BD & Month & - & 2.74 (5.7e-1, 5.0) & 1.41 (7.4e-1, 2.7)\\
    \textit{M. tuberculosis} & BD & Year & - & 2.66 (4.6e-1, 5.1) & 1.53 (8.1e-1, 2.9)\\
    \bottomrule
    \end{tabular}
\end{table}


\end{document}